\documentclass[12pt, a4paper]{report}
\usepackage[utf8]{inputenc}
\newcommand\preamble{
    \usepackage[italian]{babel}
    \usepackage{geometry}
    \usepackage{amsmath}
    \usepackage{amssymb}
    \usepackage{graphicx}
    \usepackage{ulem}
    \geometry{margin=2cm}
    \usepackage{listings}
    \usepackage{titling}
    \let\olditemize\itemize
    \renewcommand\itemize{\olditemize\setlength\itemsep{0em}}
}
% Definizione delle variabili
\newcommand{\imagePath}{Immagini/logoUni.png}

% Definizione del comando per la pagina di titolo con argomenti
\newcommand{\customTitlePage}[5]{
    \newcommand{\courseTitle}{#1}
    \newcommand{\authorName}{#2}
    \newcommand{\academicYear}{#3}
    \newcommand{\universityName}{#4}
    
    \begin{titlepage}
        \centering
        \includegraphics[width=0.5\textwidth]{\imagePath}\par\vspace{1cm}
        {\scshape\LARGE \universityName \par}
        \vspace{1.5cm}
        {\huge\bfseries \courseTitle \par}
        \vspace{2cm}
        {\Large\itshape \authorName \par}
        \vfill
        \academicYear
    \end{titlepage}
}
  
\preamble

\begin{document}
\customTitlePage{Fondamenti dell'Elaborazione di Segnali e Immagini}{Lorenzo Vaccarecci}{Anno Accademico 2024/2025}{Università degli Studi di Genova}
\newpage
\tableofcontents
\chapter{Introduzione}
\section{Segnali 1D e 2D}
\subsection{Segnali 1D}
Un segnale 1D descrive una grandezza fisica che varia nel tempo, e può essere visto come una funzione di una variabile indipendente: 
\begin{equation*}
    g = f(t)
\end{equation*}
dove $g$ è il valore della grandezza fisica (variabile \textbf{dipendente}), $f$ è la funzione (continua o discreta) e $t$ è la variabile indipendente.\\
Esempi di segnali 1D sono:
\begin{itemize}
    \item Segnali audio: come ad esempio la musica o il parlato.
    \item Segnali ECG
    \item Segnali EEG
    \item Sensori inerziali
    \item \dots
\end{itemize}
\subsection{Segnali 2D}
Un segnale 2D descrive una grandezza fisica che varia nello spazio, e può essere visto come una funzione di due variabili indipendenti.\\
Esempi di segnali 2D sono:
\begin{itemize}
    \item Immagini: utilizzeremo questo termine per indicare una foto a colori o a scala di grigi (ci concentreremo su queste).
    \item Immagini biomediche: come ad esempio le radiografie, le ecografie oppure quelle di una risonanza.
    \item Immagini termiche
    \item Immagini satellitari
    \item Immagini microscopiche
    \item \dots
\end{itemize}
Ciò che hanno in comunque tutte queste immagini è che hanno una matrice di pixel che rappresenta qualcosa, nel nostro caso ogni pixel rappresenta l'intensità luminosa nella posizione $(r,c)$ della matrice.
\section{Segnali a tempo continuo o discreto}
\subsection{Segnali a tempo continuo}
Un segnale a tempo continuo è un segnale in cui la variabile indipendente (tempo) può assumere qualsiasi valore in un intervallo continuo.
\subsection{Segnali a tempo discreto}
Un segnale a tempo discreto è un segnale in cui la variabile indipendente (tempo) può assumere solo valori discreti.
\end{document}